%!TEX root = ../main.tex

\chapter{Exercise session 18/09} % (fold)
\label{cha:exercise_session_18_09}
\thispagestyle{empty}

Today's \textbf{aim}: we want to get the essence of the notion of "being closed" in order to deal with continuity (so this lesson will be a little more theoretical than the following ones).

\section{Recall on \texorpdfstring{$\RR^n$}{C}} % (fold)
\label{sec:recall_on_rr^n_c_}

Given $x,y\in\RR$ a possible distance between $x$ and $y$ is
\begin{equation*}
    d(x,y):=\abs{x-y}
\end{equation*}

(we will analyze its properties in a moment)

Given $x,y\in\RR^2$ a possible distance between $x=(x_1,x_2)$ and $y=(y_1,y_2)$ is
\begin{equation*}
    d_E(x,y):=\sqrt{\sum_{i=1}^2\abs{x_i-y_i}^2}
\end{equation*}

the \textbf{Euclidian/canonical distance}. For $\RR^n$ is just the same.

Given $f,g\in\Cc^0\td{[a,b]}$, i.e. $f,g:[a,b]\to\RR$ continuous, the distance
\begin{equation*}
    d(f,g):=\min_{x\in[a,b]} \abs{f(x)-g(x)}
\end{equation*}

cannot be a proper distance because $d=0 \notimplies f=g$ (see the definition of distance in the next page).

\fg{0.4}{screen1809}

% section recall_on_rr^n_c_ (end)








% chapter exercise_session_18_09 (end)