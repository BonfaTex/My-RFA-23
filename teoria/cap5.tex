%!TEX root = ../main.tex

\chapter{The Lebesgue Integral}
\thispagestyle{empty}

The Lebesgue integral is built on three steps.
\begin{itemize}
    \item First, integral of simple measurable nonnegative functions
    \item Second, integral of measurable nonnegative functions
    \item Finally, integral of measurable sign-changing functions
\end{itemize}

\section{Integral of nonnegative simple functions}
\begin{defn}$\\$
Let $(X,\Mdu,\mu)$ be a measure space and let \(s: X \to [0, \infty]\) be a measurable simple function, 
\[
    s(x) = \sum_{i=1}^k a_i \Xc_{D_i}
\]
where $0\leq a_i\leq\infty$, \(D_1,\ldots,D_k\in\Mdu\), disjoint, and \(\bigcup_{i=1}^k D_i = X\). Let also \(E \in \Mdu\). Then we define 
\begin{equation}
\label{int_of_nn_simple}
\int_E s \demu := \sum_{i=1}^k a_i\, \mu(D_i \cap E)
\end{equation}
as the (Lebesgue) integral of $s$ over $E$ (w.r.t. $\mu$).
\end{defn}

Let's visualize \eqref{int_of_nn_simple}:
\fg{0.9}{SCR-20240204-nbfy}

\begin{rem}\leavevmode
\begin{enumerate}
    \item Given a simple function \(s\):
    \[s:[a,b] \to \RR,\ \lambda = \mu \qquad\Longrightarrow\qquad \int_E s \demu\ \mbox{ is the area below the graph}
    \]

    \item $D\in\Mdu\ \Longrightarrow\ \Xc_D$ is simple, indeed $\Xc_D=1\cdot\Xc_D+0\cdot\Xc_{X\setminus D}$ thus
    \begin{equation*}
    \int_X \Xc_D\demu = 1\cdot\mu(D)+0\cdot\mu(X\setminus D)=\mu(D)
    \end{equation*}

    \textit{In particular}: $\int_\RR \Xc_\QQ\dela=\lambda(\QQ)=0$

    \item More generally: $s$ simple and measurable, $E\in\Mdu$, then
    \begin{equation*}
        \int_E s\demu=\int_X s\cdot\Xc_E\demu
    \end{equation*}
\end{enumerate}
\end{rem}

\begin{thm}[Basic properties]$\\$
$(X,\Mdu,\mu)$ measure space, $N,E,F\in\Mdu$, $s_1,s_2:X\to[0,\infty]$ simple measurable. Then:
\begin{equation*}
\begin{aligned}
&
\left.
{\renewcommand*{\arraystretch}{2}
\begin{array}{ll}
1) & \mu(N)=0\qquad\Longrightarrow\qquad\displaystyle\int_N s_1\demu=0
\end{array}}
\right.
\\
&
\left.
{\renewcommand*{\arraystretch}{2}
\begin{array}{ll}
2) & 0\leq c\leq\infty \qquad\Longrightarrow\qquad\displaystyle \int_E cs_1\demu=c\int_E s_1\demu \\
3) & \displaystyle \int_E \left( s_1+s_2 \right)\demu=\int_E s_1\demu+\int_E s_2\demu
\end{array}}
\right]\qquad\text{linearity}
\\
&
\left.
{\renewcommand*{\arraystretch}{2}
\begin{array}{ll}
4) & s_1\leq s_2 \qquad\Longrightarrow\qquad\displaystyle \int_E s_1\demu\leq \int_E s_2\demu \\
5) & E\subset F \qquad\Longrightarrow\qquad\displaystyle \int_E s_1\demu\leq \int_F s_2\demu
\end{array}}
\right]\qquad\text{monotonicity}
\end{aligned}
\end{equation*}
\end{thm}

\begin{thm}[Measure defined by the integral of a nonnegative simple function]$\\$
\label{mdbioansf}
$(X,\Mdu,\mu)$ measure space, $s:X\to[0,\infty]$ simple measurable. Define 
\begin{equation*}
    \begin{aligned}
        \varphi:\Mdu&\rightarrow [0,\infty] \\
        E&\mapsto\varphi(E)=\int_E s\demu
    \end{aligned}
\end{equation*}
Then $\varphi$ is a measure on $(X,\Mdu)$.
\end{thm}

\begin{proof}[\textcolor{red}{$\star$}]
Let's check the measure definition for $\varphi$:
\begin{enumerate}
\item $\varphi(E)\geq 0\ \forall E\in\Mdu$ because $s\geq 0$

\item $\varphi(\varnothing)=\int_\varnothing s\demu =0$ because $\mu(\varnothing)=0$

\item Take $\left\{ E_n \right\}_{n\in\NN}\subset\Mdu$ pairwise disjoint and call $E=\bigcup_n E_n\in\Mdu$. Then
\begin{gather*}
\varphi(E)=\int_E s\demu = \sum_{i=1}^k a_i\,\mu\left(D_i\cap E\right) = \sum_{i=1}^k a_i\,\mu \Bigg( \bigcup_n \underbracket[0.5pt]{D_i\cap E_n}_{\text{disjoint}}\Bigg) \\
\overset{\mu\ \sigma-\text{add}}{=} \sum_{i=1}^k a_i \Bigg( \sum_n \mu\left( D_i\cap E_n \right) \Bigg) = \sum_n \sum_{i=1}^k a_i\,\mu\left( D_i\cap E_n \right)=\sum_n \varphi(E_n)
\end{gather*} 
\end{enumerate}
\end{proof}

\section{Integral of nonnegative measurable functions}

\begin{defn}$\\$
$(X,\Mdu,\mu)$ measure space, \(f:X \to [0, \infty]\) measurable, \(E \in \Mdu\). The Lebesgue integral of \(f\) on \(E\)  (w.r.t. \(\mu\)), is 
\begin{equation}
\label{int_of_nn_meas}
\int_E f \demu := \sup \left\{ \int_E s \demu \; \bigg\vert \begin{array}{c}s\text{ simple, meas.} \\ 0 \leq s \leq f \end{array}\right\}
\end{equation}
\end{defn}

\begin{rem}\leavevmode
\begin{enumerate}
    \item If \(f \) is simple, the definitions \eqref{int_of_nn_simple} and \eqref{int_of_nn_meas} are consistent

    \item \( \left( \NN, \Peu(\NN), \mu_\# \right)\). Then \(f: \NN \to [0,\infty]\) is a sequence \( \left\{ a_n \right\}_{n \in \NN}\) \[ \int_\NN \{a_n\} \demu_\# = \sum_{n\in\NN} a_n\]

    \item All the basic properties about $\displaystyle\int_E s\demu$ are true also for $\displaystyle\int_E f\demu\qquad\qquad$ \danger
\end{enumerate}   
\end{rem}

\section{Chebychev's inequality}

\begin{flushright}
\textit{From now on we always consider a complete measure space,} \\
\textit{even if without completness everything still hold.}
\end{flushright}

\begin{thm}[Chebychev's inequality]$\\$
$(X,\Mdu,\mu)$ complete measure space, \(f: X \to \left[0, \infty \right]\) measurable, \(0<c<\infty\). Then 
\[ 
\mu\big(\{f \geq c \}\big) \leq \frac{1}{c} \int_{\{f \geq c \}} f \demu \leq \frac{1}{c} \int_X f \demu 
\]

(\textit{remember that} $\{f \geq c \}$ \textit{stands for} $\{x\in X\,:\,f(x)\geq c\}$)
\end{thm}

\begin{proof}[\textcolor{red}{$\star$}]$\\$
Just apply the monotonicity property of the intergral two times
\[ \int_X f \demu \overset{X \supset \{f \geq c\}}{\geq} \int_{\{f \geq c \}} f \demu \geq \int_{\{f \geq c\}} c \demu 
= c \int_{\{f \geq c\}} \demu 
= c \mu \big(\{f \geq c\}\big) \]
and then divide by $c$.
\end{proof}

\section{Consequences of Chebychev's inequality}

\begin{lemma}[Vanishing Lemma]$\\$
$(X,\Mdu,\mu)$ complete measure space, \(f: X \to \left[0, \infty\right]\) measurable, $E\in\Mdu$. Then
\[
\int_E f \demu =0\qquad \Longleftrightarrow\qquad f=0\ \text{ for a.e. } x\in E 
\]
\end{lemma}

\begin{proof}[\textcolor{red}{$\star$}]\leavevmode
\begin{itemize}
\item[($\Leftarrow)$] easy (exercise)

\item[($\Rightarrow$)] We have to show that
\begin{equation*}
\mu\left( \{x\in E\,:\,f(x)> 0\}  \right)=0
\end{equation*}

Consider \( E \cap \{f >0\} = \displaystyle\bigcup_{n=1}^\infty \underbrace{\left(E \cap \left\{ f \geq \frac{1}{n} \right\} \right)}_{=:E_n} \) 

Then \(\{E_n\}\) is an increasing sequence, hence $\displaystyle \mu(E\cap\{f>0\})=\lim_n \mu(E_n)$ for the continuity of the measure along monotone sequences. By Chebychev 
\[
    0\leq \mu (E_n) \leq \frac{1}{\nicefrac{1}{n}} \underbracket[0.5pt]{\int_{E_n} f \demu}_{=0} =0 \quad \forall n \qquad  \Longrightarrow\qquad \mu(E_n)=0 \quad \forall n 
\]
Thus \(\mu(E \cup \{f>0\})= \displaystyle \lim_n \mu (E_n)=0\), namely \(f=0\) a.e. on \(E\).
\end{itemize}
\end{proof}

\begin{marker} 
The statement is trivial when $\mu(E)=0$: \emph{any pointwise property is true a.e. on a set of zero measure} \\ $\leadsto$ \textbf{the integral \textit{does not see} sets of zero measure}.    
\end{marker}

\begin{lemma}[Integrable functions are a.e. finite]$\\$
$(X,\Mdu,\mu)$ complete measure space, \(f: X \to \left[0, \infty\right]\) measurable s.t. $\displaystyle\int_X f\demu <\infty$ (i.e. $f$ is integrable). Then
$$
\mu\big( \{x\in X\,:\, f(x)=\infty \} \big)=0
$$
\end{lemma}

\begin{proof}$\\$
Exercise (hint: $\{f=\infty\}=\bigcap_n\{f\geq n\}$, then apply Chebychev + continuity).
\end{proof}

\newpage

\section{Monotone Convergence Theorem}

\begin{thm}[Monotone Convergence (MCT) - Beppo Levi]$\\$
$(X,\Mdu,\mu)$ complete measure space, \(f_n:X\to \left[0, \infty\right]\) measurable $\forall n\in\NN$. Suppose that:
\begin{itemize}
    \item[(i)] \(f_n(x) \leq f_{n+1}(x)\) for a.e. \(x \in X\), for every \(n\in\NN\)
    \item[(ii)] \(f_n(x) \to f(x) \) for a.e. \(x\in X\)
\end{itemize} 

Then 
\[ 
\int_X f \demu = \lim_n \int_X f_n \demu
\]

(\textit{posso portar dentro il limite})
\end{thm}

\begin{proof}[\textcolor{red}{$\star$}]\leavevmode
\begin{enumerate}
\item Assume that the two hypothesis hold $\forall x\in X$. 

First, $f=\lim_n f_n$ thus $f\geq0$ and measurable, and
\[
    \alpha_n=\int_X f_n \demu \nearrow \qquad \Longrightarrow \qquad \exists \alpha = \lim_n \int_X f_n \demu\in[0,+\infty]
\]

Also, \(f_n \leq f \) everywhere so \(\displaystyle \int_X f_n \demu \overset{\text{mon.}}{\leq} \int_X f \demu \quad \forall n\) 
\[
    \Longrightarrow\qquad \alpha \leq \int_X f \demu 
\]

We want to show that also \(\geq \) is true. 

Take any simple, meas. \(s:X\to[0,\infty]\)  s.t. \(0 \leq s \leq f\) in $X$ and take \(c \in \left(0,1\right)\). Consider $f_n(x)\nearrow f(x)\ \forall x\in X$ and the sets $E_n =\{x\in X\,:\,f_n(x)\geq c\cdot s(x)\}=\{f_n \geq cs\} \in \Mdu$. We have:
\begin{itemize}
    \item \(E_n \subset E_{n+1} \ \forall \ n,\text{ indeed}:\) 
    \\ if \(x \in E_n, \) then \(f_n(x) \geq cs(x) \Rightarrow f_{n+1}(x) \geq cs(x) \Rightarrow x \in E_{n+1}\)
    \item \(X = \displaystyle\bigcup_{n=1}^\infty E_n\), indeed: 
    \\ - if \(f(x)=0\), then \(x \in E_1\) 
    \\ - if \(f(x)>0\), then \(cs(x) < f(x)=\lim_n f_n(x)\) since \(s \leq f \) and \(c <1\) 
    \\ \(\quad\Rightarrow cs(x) < f_n(x)\) for \(n \) sufficiently large, namely \(x \in E_n \) for \(n \) sufficiently large
\end{itemize} 

Therefore
\[
    \alpha \geq \int_X f_n \demu \geq \int_{E_n} f_n \demu \geq c \int_{E_n} s \demu = c \varphi(E_n) \qquad \forall n
\]

Since \(\varphi\) is a measure (thm. \ref{mdbioansf}), and \(\{E_n\} \nearrow\) we can take the limit when \(n \to \infty\) by continuity
\[
    \alpha \geq c\varphi\left(\lim_n E_n\right)=c\varphi(X)=c\int_X s\demu \qquad \forall\ 0\leq s \leq f \text{ simple and meas., and }\forall\ 0<c<1
\]

Finally, take the limit when \(c \to 1: \ \alpha \geq \int_X s \demu \), and take the sup over $s$: \(\alpha \geq \int_X f \demu \).

\item \(F=\{x \in X: \text{either (i) or (ii) fails}\}\) is a set of zero measure. Take $E=X\setminus F$, then (i) and (ii) are true $\forall x\in E$, and $X=F\cup E$. By the Vanishing Lemma, since \(\left(f - f\cdot\Xc_E\right)=0\) a.e. in $X$ we have 
\[ 
\int_X \left(f - f\cdot\Xc_E\right)\demu=0\qquad\Longrightarrow\qquad
\int_X f \demu = \int_X f\cdot\Xc_E\demu= \int_E f \demu \overset{1.}{=} \lim_n \int_E f_n \demu = \lim_n \int_X f_n \demu 
\]
\end{enumerate}    
\end{proof}

\newpage

\section{Consequences of Monotone Convergence Theorem}

\begin{coro}[Monotone convergence for series]$\\$
$(X,\Mdu,\mu)$ complete measure space, \(f_n : X \to [0, \infty]\) measurable $\forall n\in\NN$, then 
    \[
        \int_X \left( \sum_{n=1}^{\infty} f_n\right) \demu = \sum_{n=1}^{\infty}\left( \int_X f_n \demu \right)
    \]
(\textit{posso scambiare serie e integrale})
\end{coro}

\begin{proof}
Exercise (apply MCT to $g_n=\sum_{k=0}^n f_n$).
\end{proof}

\begin{rem}
Once, for $f:X\to[0,\infty]$ measurable we had
\[
\int_X f \demu = \sup \left\{ \int_X s \demu \; \bigg\vert \begin{array}{c}s\text{ simple, meas.} \\ 0 \leq s \leq f \end{array}\right\}
\]

But now, let \(\left\{ s_n \right\}\) be the sequence given by the simple approximation theorem. By MCT
\[
    \int_X f \demu = \lim_n \int_X s_n \demu
\]

much simplier (sometimes).

\begin{exa}
\(f, g : X \to [0, \infty]\). Then 
\[
    \int_X (f+g) \demu = \int_X f \demu + \int_X g \demu
\]
\end{exa}
\end{rem}

\newpage

\begin{thm}[Measure defined by the integral of a nonnegative measurable function]$\\$
\label{mdbioanmf}
\((X, \Mdu, \mu)\) complete measure space, \(\Phi : X \to [0, \infty]\) measurable, $E\in\Mdu$. Define
\begin{equation*}
\nu(E) = \int_E \Phi \demu=\int_X\Phi\Xc_E\demu
\end{equation*}

Then \(\nu : \Mdu \to [0, +\infty]\) is a measure. Moreover, let \(f:X \to [0, \infty]\) measurable. Then
\begin{equation}
\label{sigma-add-for-int}
\int_X f \denu = \int_X f\Phi \demu
\end{equation}

(\textit{you may think} "$\denu=\Phi\demu$")
\end{thm}

\begin{proof}[\textcolor{red}{$\star$}]\leavevmode
\begin{enumerate}
\item $\nu$ it's a well-defined measure because $\Phi\geq 0$
\item $\nu(\varnothing)=0$ because $\mu(\varnothing)=0$
\item $\sigma$-addditivity? Take $\{E_n\}_n\subset\Mdu$ disjoint and call $E=\bigcup_n E_n$. Then
\begin{equation*}
\begin{aligned}
\nu(E)&=\int_X\Phi\Xc_E\demu=\int_X\Phi\sum_n\Xc_{E_n} \demu=\int_X \left( \sum_n\Phi\Xc_{E_n} \right)\demu \\
&\overset{\text{MCT x }\Sigma}{=} \sum_n \left( \int_X\Phi\Xc_{E_n}\demu \right) = \sum_n\nu(E_n)
\end{aligned}
\end{equation*}
$\leadsto$ $\nu$ is a measure

\item We want to show \eqref{sigma-add-for-int} for $f=\Xc_F$, $F\in\Mdu$ (in this way you can generalize thanks to the simple approx. thm.). We have
\begin{equation*}
\int_X\Xc_F\denu=\int_F 1\denu = \nu(F)=\int_X\Xc_F\Phi\demu
\end{equation*}
\end{enumerate}
\end{proof}

\begin{lemma}[Fatou's Lemma]$\\$
\((X, \Mdu, \mu)\) complete measure space, \(f_n: X \to [0, \infty]\) measurable \(\forall n\). Then 
    \[
        \int_X \left(\liminf_{n\to\infty} f_n\right) \demu \leq \liminf_{n\to\infty} \left(\int_X f_n \demu\right)
    \]
\end{lemma}

\begin{proof}[\textcolor{red}{$\star$}]$\\$
We know that \((\liminf_n f_n)(x) = \sup_n \Big(\underbrace{\inf_{k \geq n} f_k(x)}_{= g_n (x)}\Big)\). For every \(x \in X\): 
\begin{equation*}
(a)\ g_n(x)\leq g_{n+1}(x)\qquad\text{and}\qquad (b)\ g_n(x)\leq f_n(x)
\end{equation*}

Thus, $(a)$ + def. of $g_n$ $\Longrightarrow$ we can apply MCT to $g_n$:
\[
\int_X \left( \liminf_n f_n \right) \demu = \int_X \left(\lim_n g_n\right) \demu \overset{\text{MCT}}{=} \lim_n \left( \int_X g_n \demu \right) = \liminf_n \left( \int_X g_n \demu \right) \overset{(b)}{\leq}\liminf_n \left(\int_X f_n\demu\right)
\]
\end{proof}

\newpage

\section{Integrable functions}

\begin{flushright}
\textit{Finally, the Lebesgue integral for functions without sign restrictions.}
\end{flushright}

\begin{defn}[Integrable functions and the Set of integrable functions]$\\$
\((X, \mathcal{M}, \mu)\) complete measure space, then \(f: X \to \overline{\RR}\) measurable is integrable on \(X\) if
    \[
        (0\leq)\quad \int_X |f| \demu < \infty
    \]
Moreover, we define the set of (measurable and) integrable functions as
\begin{equation*}
\rsf{L}^1(X,\Mdu,\mu):=\left\{f:X\to\overline{\RR}\text{ meas. s.t. }\int_X |f| \demu < \infty\right\}
\end{equation*}
\end{defn}

If \(f\) is integrable we define its integral as
\begin{equation*}
\int_X f\demu = \int_X f^+\demu-\int_X f^-\demu
\end{equation*}

because we know that $f$ measurable $\ \Rightarrow\ $ $f=f^+-f^-$ with $f^\pm\geq0$ and measurable, and we've already defined the integral for nonnegative measurable functions. 

With the same spirit, $|f|=f^++f^-$ thus
\begin{equation*}
\int_X |f|\demu = \int_X f^+\demu+\int_X f^-\demu
\end{equation*}

and if $E\in\Mdu$
\begin{equation*}
\int_E f\demu=\int_X f\cdot\Xc_E\demu
\end{equation*}

\begin{proposition}$\\$
\((X, \Mdu, \mu)\) complete measure space, \(f: X \to \overline{\RR}\) measurable. Then
\begin{enumerate}[(i)]
    \item $f\in \rsf{L}^1\ \Longleftrightarrow\ |f|\in\rsf{L}^1\ \Longleftrightarrow\ f^\pm\in\rsf{L}^1$
    \item \textbf{triangle inequality}: $f\in \rsf{L}^1$ then
    \begin{equation*}
    (0\leq)\quad \left| \int_X f\demu \right| \leq \int_X |f|\demu 
    \end{equation*}
    (\textit{posso portar dentro il modulo})
\end{enumerate}
\end{proposition}

\begin{proof}[\textcolor{red}{$\star$}]
Only (ii):
\begin{align*}
\left| \int_X f\demu \right|&=\left| \int_X f^+\demu-\int_X f^-\demu \right| \\
&\leq \left| \int_Xf^+\demu \right|+\left| \int_X f^-\demu \right| \\
&=\int_X f^+\demu+\int_X f^-\demu\\
&=\int_X \left( f^++f^-  \right)\demu\\
&=\int_X |f|\demu
\end{align*}
\end{proof}

\begin{proposition}$\\$
\((X, \Mdu, \mu)\) complete measure space. Then $\rsf{L}^1(X, \Mdu, \mu)$ is a vector space, and $\int\cdot\demu$ is linear:
\begin{gather*}
f,g: X \to \overline{\RR}\in\rsf{L}^1,\quad \alpha,\beta\in\RR \\
\Downarrow \\
\alpha f+\beta g\in\rsf{L}^1\quad\text{and}\quad \int_X (\alpha f+\beta g)\demu=\alpha\int_Xf\demu+\beta\int_X g\demu
\end{gather*}
\end{proposition}

All other results can be extended from nonnegative functions to sign-changing functions. For examples:
\begin{lemma}[Vanishing Lemma, turbo]$\\$
\((X, \Mdu, \mu)\) complete measure space, \(f,g: X \to \overline{\RR}\in\rsf{L}^1\). Then
\[
    f=g \text{ a.e. on } X\quad \Longleftrightarrow\quad \int_X |f-g| \demu =0 \quad\Longleftrightarrow\quad \int_E f \demu = \int_E g \demu \quad \forall E \in \Mdu 
\] 
\end{lemma}

\section{Dominated convergence theorem}

The most relevant theorem for convergence is the following:

\begin{thm}[Dominated convergence theorem]$\\$
\((X, \Mdu, \mu)\) complete measure space, \(\{f_n\}\) sequence of measurable functions \(f_n: X \to \overline{\RR}\). Assume:
    \begin{enumerate}[(i)]
        \item \(f_n(x) \to f(x) \) for a.e. $x\in X$
        \item \(\exists\) a nonnegative function \(g \in \rsf{L}^1(X)\) such that
        \begin{equation*}
        \left| f_n(x) \right| \leq g(x)\quad \text{for a.e. }x\in X,\ \forall n\in\NN
        \end{equation*}
        ($g$ \emph{dominating function})
    \end{enumerate}

    Then \(f \in \rsf{L}^1(X)\) and 
    \begin{equation}
    \label{dct}
    \boxed{\lim_n \int_X \left|f_n -f\right| \demu = 0  } 
    \end{equation}

    In particular,
    \begin{equation}
    \label{dct-impl}
    \lim_n \int_X f_n \demu = \int_X f \demu
    \end{equation}
    (\textit{posso portar dentro il limite})
\end{thm}

\begin{marker} 
Note that \eqref{dct}, i.e.
\begin{equation*}
\int_X \left|f_n -f\right| \demu \xrightarrow{\quad}0
\end{equation*}
tells you more than \eqref{dct-impl}, i.e.
\begin{equation*}
\int_X f_n \demu \xrightarrow{\quad} \int_X f \demu
\end{equation*}

Indeed, it introduces a concept of "distance". We'll see this soon.
\end{marker}

\begin{rem}$\\$
If \(\mu(X) < \infty\), then costants are integrable. So if $\forall n$ there exists  \(M > 0\) s.t. \(\abs{f_n} \leq M\) a.e. on \(X\), we just take \(g(x) = M\) as dominating function. 
\end{rem}

\newpage

\begin{proof}[\textcolor{red}{$\star$}]$\\$
Since $f_n$ are measurable then $|f_n|$ are nonnegative and measurable, thus for the monotonicity of the integral for nonnegative measurable functions we have
\begin{equation*}
\int_X |f_n|\demu\leq \int_X g\demu <\infty\quad\text{'cause }g\in\rsf{L}^1(X)
\end{equation*}

i.e. $f_n\in\rsf{L}^1(X)$. In addiction
\begin{equation*}
\left.
\begin{gathered}
f_n\text{ meas.} \\
f_n\to f\text{ a.e.}
\end{gathered}\right\}\qquad\Longrightarrow\qquad f\text{ meas.}
\end{equation*}

thus $|f|$ is nonnegative measurable and since $|f_n|\leq g$ a.e. for $n\to\infty$ we have $|f|\leq g$ a.e., so also $f\in\rsf{L}^1(X)$ for monotonicity.

Define an auxiliar sequence $\{ h_n\}_n$ s.t.
\begin{equation*}
h_n(x)=2g(x)-\underbracket[0.5pt]{|f_n(x)-f(x)|}_{\text{what I want to eval}}
\end{equation*}

Check the Fatou's Lemma assumption:
\begin{itemize}
    \item $h_n$ meas. because is a sum (cotninuous operation) of meas. functions
    \item $h_n$ nonnegative because
    \begin{gather*}
    |f_n-f|\leq |f_n|+|f|\leq g+g=2g \\
    \Longrightarrow\quad 2g-|f_n-f|=h_n\geq 0
    \end{gather*}
\end{itemize}

So we can apply Fatou to $\{h_n\}$:
\begin{gather*}
\begin{gathered}
\int_X \left( \liminf_n h_n  \right) \demu \leq \liminf_n \left( \int_X h_n\demu \right) \\
\int_X \left[ \liminf_n \big( 2g-|f_n-f| \big)  \right] \demu \leq \liminf_n \left( \int_X 2g\demu -\int_X |f_n-f|\demu \right) \\
\cancel{\int_X 2g\demu} \leq \cancel{\int_X 2g\demu} + \liminf_n \left( -\int_X |f_n-f|\demu \right)  
\end{gathered} \\
\overset{\circledast}{\Longrightarrow}\qquad 0\geq \limsup_n \int_X |f_n-f|\demu \geq \liminf_n \int_X |f_n-f|\demu \geq 0 
\end{gather*}

hence $\exists\;\displaystyle\lim_n \int_X |f_n-f|\demu$ and its value is equal to zero.

Now we need to prove \eqref{dct-impl}:
\begin{equation*}
0\leq \left| \int_X f_n\demu-\int_X f\demu \right|\overset{\text{lin.}}{=}\left| \int_X f_n-f\demu\right|\overset{\text{tr.}}{\leq}\int_X |f_n-f|\demu\xrightarrow{\eqref{dct}}0
\end{equation*}
\end{proof}

\begin{marker}
Let's motivate $\circledast$:
\begin{gather*}
\limsup_n a_n = \inf_n \sup_{k\geq n} a_k \\
\Downarrow \\
-\limsup_n a_n = - \inf_n \sup_{k\geq n} a_k=\sup_n \left( -\sup_{k\geq n} a_k \right)=\sup_n \inf_{k\geq n} \left( - a_k \right)=\liminf_n \left( - a_n \right)
\end{gather*}
\end{marker}

\begin{coro}[Dominated convergence for series]$\\$
$(X,\Mdu,\mu)$ complete measure space, \(\{f_n\}_{n\in\NN}\subset\rsf{L}^1(X,\Mdu,\mu)\). If
\begin{equation*}
\sum_{n=1}^\infty \int_X |f_n|\demu<\infty
\end{equation*}
then
\[
    \int_X \left( \sum_{n=1}^{\infty} f_n\right) \demu = \sum_{n=1}^{\infty}\left( \int_X f_n \demu \right)
\]
(\textit{posso scambiare serie e integrale})
\end{coro}

\section{Comparison between Riemann and Lebesgue integrals}

\underline{Recall}: 
\begin{itemize}
    \item $f$ is Riemann-integrable iff $f$ is continuous for a.e. $x\in[a,b]$. In particular, $\Xc_\QQ$ is not Riemann-integrable but, as we saw at the beginning of this chapter, its Lebesgue-integrable is zero.

    (\textit{Lebesgue}$\not\Rightarrow$\textit{Riemann})

    \item $f$ Riemann-integrable in the generalized sense means Riemann-integrable for bounded functions on unbounded domains or Riemann-integrable for unbounded functions on bounded domains.
\end{itemize}

\begin{thm}$\\$
Let $I=[a,b]\subset\RR$. Assume that $f:I\to\RR$ is Riemann-integrable. Then  
\[
    f\in \rsf{L}^1(I, \rsf{L}(I), \lambda)
\]
and 
\[
    \int_{[a,b]} f \dela = \int_{[a,b]} f(x) \dx
\]

(\textit{Riemann}$\Rightarrow$\textit{Lebesgue})
\end{thm}

\begin{thm}$\\$
$\alpha,\beta\in\overline{\RR}$, $I=(\alpha,\beta)$. If $|f|$ is Riemann-integrable in the generalized sense then
\begin{equation*}
f\in\rsf{L}^1(I, \rsf{L}(I), \lambda)\qquad\text{and}\qquad \int_{(\alpha,\beta)}f\dela=\int_\alpha^\beta f(x)\dx
\end{equation*}
\end{thm}

\danger: it's not true that always \textit{Riemann}$\Rightarrow$\textit{Lebesgue}. For istance, if
\begin{itemize}
    \item $|f|$ is \underline{not} Riemann-integrable in the generalized sense

    \item $f$ is Riemann-integrable in the generalized sense
\end{itemize}
then the Riemann and Lebesgue integrals are not related.

\begin{exa}$\\$
\(f:(0,\infty)\to\RR,\ f(x) = \dfrac{\sin x}{x},\ \displaystyle \int_0^{\infty} \vert f(x) \vert \dx = +\infty\ \Longrightarrow\ f \not \in \rsf{L}^1((0,\infty)) \) \\
but on the other hand for the Riemann integral: $\displaystyle\int_0^{\infty} \frac{\sin x}{x} \dx = \lim_{\omega \to \infty} \int_0^{\omega} \frac{\sin x}{x} \dx = \frac{\pi}{2}$

(\textit{the problem is that this function changes sign infinitely many times})
\end{exa}











