%!TEX root = ../main.tex

\chapter{Measure Theory} % (fold)
\label{cha:chapter_name}
\thispagestyle{empty}

\section{Measure spaces} % (fold)
\label{sec:measure_spaces}

Let $X$ be a set.

\begin{defn}[$\sigma$-algebras]$\\$
A family $\Mc \subset \Pc (x)$ is called a $\sigma$-algebra if 
\begin{enumerate}
    \item[i)] $\emptyset \in \Mc$
    \item[ii)] $ E \subset \Mc \ \Longrightarrow \ E^c  = X\setminus E \in \Mc$
    \item[iii)] $\gr{E_n}_{n \in \mathbb{N}} \subset \Mc \Longrightarrow \bigcup_{n=1}^{\infty} E_n \in \Mc $ (infinite countable union)
\end{enumerate}
If (iii) is replaced by "$E_1, E_2 \in \Mc \Longrightarrow E_1 \bigcup E_2 \in \Mc$" then $\Mc$ is just an algebra (finite union).
\end{defn}

Trivial examples: $\Mc=\Pc(X)$ is the biggest $\sigma$-algebra, $\Mc=\gr{\varnothing,X}$ is the smallest $\sigma$-algebra.

We say that
\begin{itemize}
    \item $\Mc$ $\sigma$-algebra $\leadsto$ $(X,\Mc)$ is a \textbf{measurable space}
    \item $E\in\Mc$ are \textbf{measurable sets}
\end{itemize}

Basic properties of $\Mc$:
\begin{enumerate}
    \item $X=\varnothing^c\in\Mc$ (by (i)+(ii))
    \item $\Mc$ is an algebra ($\sigma$-alg. $\Longrightarrow$ alg. but not the viceversa)

    To prove this you can take a finite union (e.g. $E_1\cup E_2$) and then make infinite unions with $\varnothing$ to have an infinite union that still belongs to $\Mc$:
    \begin{equation*}
        E_1\cup E_2=\underbracket[0.5pt]{E_1\cup E_2\underbracket[0.5pt]{\cup\varnothing\cup ... \cup\varnothing\cup...}_{\in\Mc\text{ by (i)}}}_{\in\Mc\text{ by (iii)}}
    \end{equation*}

    \item $\gr{E_n}_n\subset\Mc\Longrightarrow \bigcap_{n\in\NN} E_n\in\Mc$
    \item $E,F\in\Mc\Longrightarrow E\setminus F \in \Mc$
\end{enumerate}

Now, we want to understand how to generate a $\sigma$-algebra.

\begin{thm}
    Take $\Sc\subset\Pc(X)$ any family. Then it is well defined $\sigma_0(\Sc)$, the $\sigma$-algebra generated by $\Sc$ (the smallest $\sigma$-algebra containing $\Sc$):
    \begin{itemize}
        \item[i)] $\sigma_0(\Sc)$ is a $\sigma$-algebra
        \item[ii)] $\Sc\subset\sigma_0(\Sc)$
        \item[iii)] if $\Mc$ is a $\sigma$-alg. and $\Sc\subset\Mc$ then $\sigma_0(\Sc)\subset\Mc$
    \end{itemize}
\end{thm}

\begin{proof}[Sketch]$\\$
We introduce a collection of collection of sets (we should be more strict)
\begin{equation*}
    \Vc=\left\{\Mc\subset\Pc(X)\,:\,\Mc\text{ is a }\sigma\text{-alg. and }\Sc\subset\Mc  \right\}
\end{equation*}
(notice that $\Vc$ is not empty since $\Pc(X)\in\Vc$)

Then $\sigma_0(\Sc)=\bigcap\gr{\Mc\,:\,\Mc\in\Vc}$ (to generate the smallest take the intersection of all).
\end{proof}

% section measure_spaces (end)

\section{Borel sets} % (fold)
\label{sec:borel_sets}

% section borel_sets (end)

\section{Measures} % (fold)
\label{sec:measures}

% section measures (end)






% chapter chapter_name (end)



