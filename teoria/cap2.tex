%!TEX root = ../main.tex

\chapter{Measure Theory} % (fold)
\label{cha:measure_theory}
\thispagestyle{empty}

\section{Measure spaces} % (fold)
\label{sec:measure_spaces}

Let $X$ be a set.

\begin{defn}[$\sigma$-algebras]$\\$
A family $\Mdu \subset \Peu (x)$ is called a $\sigma$-algebra if 
\begin{enumerate}
    \item[i)] $\emptyset \in \Mdu$
    \item[ii)] $ E \subset \Mdu \ \Longrightarrow \ E^c  = X\setminus E \in \Mdu$
    \item[iii)] $\gr{E_n}_{n \in \mathbb{N}} \subset \Mdu \Longrightarrow \bigcup_{n=1}^{\infty} E_n \in \Mdu $ (infinite countable union)
\end{enumerate}
If (iii) is replaced by "$E_1, E_2 \in \Mdu \Longrightarrow E_1 \bigcup E_2 \in \Mdu$" then $\Mdu$ is just an algebra (finite union).
\end{defn}

Trivial examples: $\Mdu=\Peu(X)$ is the biggest $\sigma$-algebra, $\Mdu=\gr{\varnothing,X}$ is the smallest $\sigma$-algebra.

We say that
\begin{itemize}
    \item $\Mdu$ $\sigma$-algebra $\leadsto$ $(X,\Mdu)$ is a \textbf{measurable space}
    \item $E\in\Mdu$ are \textbf{measurable sets}
\end{itemize}

Basic properties of $\Mdu$:
\begin{enumerate}
    \item $X=\varnothing^c\in\Mdu$ (by (i)+(ii))
    \item $\Mdu$ is an algebra ($\sigma$-alg. $\Longrightarrow$ alg. but not the viceversa)

    To prove this you can take a finite union (e.g. $E_1\cup E_2$) and then make infinite unions with $\varnothing$ to have an infinite union that still belongs to $\Mdu$:
    \begin{equation*}
        E_1\cup E_2=\underbracket[0.5pt]{E_1\cup E_2\underbracket[0.5pt]{\cup\varnothing\cup ... \cup\varnothing\cup...}_{\in\Mdu\text{ by (i)}}}_{\in\Mdu\text{ by (iii)}}
    \end{equation*}

    \item $\gr{E_n}_n\subset\Mdu\Longrightarrow \bigcap_{n\in\NN} E_n\in\Mdu$
    \item $E,F\in\Mdu\Longrightarrow E\setminus F \in \Mdu$
\end{enumerate}

Now, we want to understand how to generate a $\sigma$-algebra.

\begin{thm}
    Take $\Seu\subset\Peu(X)$ any family. Then it is well defined $\sigma_0(\Seu)$, the $\sigma$-algebra generated by $\Seu$ (the smallest $\sigma$-algebra containing $\Seu$):
    \begin{itemize}
        \item[i)] $\sigma_0(\Seu)$ is a $\sigma$-algebra
        \item[ii)] $\Seu\subset\sigma_0(\Seu)$
        \item[iii)] if $\Mdu$ is a $\sigma$-alg. and $\Seu\subset\Mdu$ then $\sigma_0(\Seu)\subset\Mdu$
    \end{itemize}
\end{thm}

\begin{proof}[Sketch]$\\$
We introduce a collection of collection of sets (we should be more strict: without knowing axiom choices we cannot properly prove this theorem, all we currently need is how to construct these $\sigma$-algebras):
\begin{equation*}
    \Vc=\left\{\Mdu\subset\Peu(X)\,:\,\Mdu\text{ is a }\sigma\text{-alg. and }\Seu\subset\Mdu  \right\}
\end{equation*}
(notice that $\Vc$ is not empty since $\Peu(X)\in\Vc$)

Then $\sigma_0(\Seu)=\bigcap\gr{\Mdu\,:\,\Mdu\in\Vc}$ (to generate the smallest take the intersection of all).
\end{proof}

% section measure_spaces (end)

\section{Borel sets} % (fold)
\label{sec:borel_sets}

We now want to define between measurable and open sets, we do this by constructing the borel $\sigma$-algebra.

\begin{defn}[Borel $\sigma$-algebras and Borel sets]$\\$
    Let $(X,d)$ be a metric space, so that open subsets of $X$ are defined (a topological space is enough) and let $\Teu = \gr{E\subset X\,: \,E \text{ is open}}$. The $\sigma$-algebra generated by $\Teu$, $\sigma_0(\Teu)$, is the Borel $\sigma$-algebra of $X$, and we write $\Bdu (X) = \sigma_0(\Teu)$. 

    Furthermore, any $E\in \Bdu (X)$ is a Borel (measurable) set.
\end{defn}

All the followings are Borel sets:
\begin{itemize}
    \item all open sets
    \item all closed sets (because the $\sigma$-algebra is closed under complements, and the complements of open sets are naturally closed sets)
    \item all countable intersections of open sets (G$\delta$-sets)  
    \item All countable unions of closed sets (F$\sigma$-sets)  
\end{itemize}

We will deal with two main cases:
\begin{itemize}
    \item real numbers $X=\mathbb{R}=(-\infty,+\infty)$
    \item \textit{extended} real numbers $X=\overline{\mathbb{R}} = [-\infty,\infty]$ 
\end{itemize}

\begin{subtle}
Defining a measure in $\overline{\mathbb{R}}$ isn't trivial, we therefore define how to extend the following operations to $\overline{\mathbb{R}}$.

\textbf{Operations}: let $a\in \mathbb{R}$. Then:
\begin{itemize}
    \item $a>0\ \Longrightarrow\ a\cdot \pm \infty = \pm \infty$
    \item $a<0\ \Longrightarrow\ a\cdot \mp \infty = \pm \infty$
    \item $a\pm \infty = \pm \infty$
    \item $0\cdot \pm \infty = 0$ (note that we are not taking any limits in this assumption, we define it this way because we want the zero function to have a null integral in an unbounded interval)
    \item $+\infty -\infty$ is not defined
\end{itemize}

\textbf{Open intervals}: let $a,b\in\RR$, $a<b$. Then:
\begin{itemize}
    \item $(a,b)$ is open
    \item $[-\infty , b)$ is open
    \item $(a,+\infty ]$ is open
\end{itemize}
\end{subtle}

\newpage

We will deal with $\Bdu(\mathbb{R}) \text{ and }  \Bdu\left(\overline{\mathbb{R}}\right)$.

\begin{marker}
Note that
\begin{align}
        \Bdu(\mathbb{R}) :&= \sigma_0 \left( \gr{\text{open sets}} \right) \nonumber \\
        &=\sigma_0 \left( \gr{\text{open intervals}} \right) \nonumber \\
        &=\sigma_0 \left( \gr{(a,+\infty)}\right) \label{borel_def_3}
\end{align}
This means we can get open sets and open intervals by taking complements, unions and intersections, starting from sets in the form of (\ref{borel_def_3}). This is very useful for proving properties of open intervals and sets. \\
The properties are generally proven more easily when the set of generators is smaller.

Moreover:
    \begin{align*}
        \Bdu(\mathbb{\overline{R}}) :&= \sigma_0 \left( \gr{\text{open sets}} \right) \\
        &=\sigma_0 \left( \gr{(a,+\infty)}\right) \\
        \Bdu(\mathbb{R}^n) :&= \sigma_0 \left( \gr{\text{open rectangles}} \right) \\
        &=\sigma_0 \left( \gr{\text{closed rectangles}} \right)
    \end{align*}
\end{marker}

% section borel_sets (end)

\section{Measures} % (fold)
\label{sec:measures}

Let $(X,\Mdu)$ be a measurable space.

\begin{defn}[Measure]$\\$
A measure on $\Mdu$ is a function 
\begin{equation*}
\mu : \Mdu \longrightarrow [0,+\infty] \quad \text{s.t.} \quad
{\renewcommand*{\arraystretch}{2.5}
\begin{array}{ll}
 \text{i)} & \mu(\varnothing) = 0 \\
 \text{ii)} & \gr{E_n}_n \subset \Mdu \text{ disjoint }\Longrightarrow\ \mu\displaystyle \left( \bigcup_{n\in \NN} E_n \right) = \sum_{n\in \NN} \mu \bigl( E_n\bigr)\qquad (\sigma\text{-additivity})
\end{array}}
\end{equation*}
\end{defn}

\begin{defn}[Measure space]$\\$
Take $X,\Mdu,\mu$ as above. Then $ (X,\Mdu,\mu)$ is a measure space.

In particular:
\begin{itemize}
    \item if $\mu(X)=1$ then $ (X,\Mdu,\mu)$ is a probability space and $\mu$ is a probability measure

    \item if $\mu(X)<+\infty$ then $\mu$ is a finite measure

    \item if $\exists \gr{E_n}_n\, :\, \mu \bigl( E_n \bigr) < +\infty $ and $ X = \displaystyle\bigcup_n E_n$ then $\mu$ is a $\sigma$-finite measure
\end{itemize}
\end{defn}

Some examples:
\begin{enumerate}
    \item[1)] for any $(X,\Mdu)$\ $\longrightarrow$ \ $\mu \bigl( E \bigr) = 0 \ \ \forall E\quad$ is the \emph{trivial measure}

    \item[2)] for any $(X,\Mdu )\  \longrightarrow$ \begin{tabular}[t]{@{}l@{}}
        $\begin{cases}
            \mu(E)=+\infty, &\ \forall E\neq\varnothing \\
            \mu(\varnothing)=0
        \end{cases}\quad$
        \end{tabular} is a measure

    \item[3)] for $(X,\Peu (X) )\  \longrightarrow$ \begin{tabular}[t]{@{}l@{}}
        $\mu_{\#}(E)  =
        \begin{cases}
            \# \gr{\text{elements of } E}, &\text{ if } E\text{ is finite} \\
            +\infty, &\ \text{otherwise}
        \end{cases}\quad$
    \end{tabular} is the \emph{counting measure}
    
    \item[4)] for $(X,\Peu (X) )$ with $X$ nonempty, pick $x_0 \in X \ \longrightarrow$ \begin{tabular}[t]{@{}l@{}}
        $\delta_{x_0}(E)  =$
        $\begin{cases}
            1, &\text{ if } x_0 \in X  \\
            0, & \text{ otherwise}
        \end{cases}\quad$
    \end{tabular} is the \emph{Dirac measure}
\end{enumerate}

% section measures (end)

\section{Properties of measures} % (fold)
\label{sec:properties-of-measures}

Our measure space will be $(X,\Mdu,\mu)$ unless otherwise stated.
\begin{thm}[Basic properties]$\\$
    \begin{itemize}
        \item[i)] $\mu \bigl( E \bigr) < + \infty ,\ \land \ (E,F) \in \Mdu, \ \land \ E \cup F = \emptyset \ \Longrightarrow \ \mu \bigl( E \cup F \bigr) = \mu \bigl( E \bigr) + \mu \bigl( F \bigr) $ (Finite additivity)
        \item[ii)] $E,F \in \Mdu \ \land \ E \subset F \ \Longrightarrow \ \mu \bigl( E \setminus F \bigr) = \mu \bigl( E \bigr) + \mu \bigl( F \bigr)$ (Excision)
        \item[iii)] $E,F \in \Mdu \ \land \ E \in F \ \Longrightarrow \ \mu \bigl( E \bigr) < \mu \bigl( F \bigr) $ (Monotonicity)
    \end{itemize}
\end{thm}
\begin{proof}[]$\\$
    \begin{itemize}
        \item[i)] Take $E_1 = E \ , \ E_2 = F \, \ E_n = \emptyset \ \forall n \geq 3$ therefore we have a disjoint family:
        \begin{align*}
            \mu \bigl( E \cup F \bigr) 
            &\equalexpl{For $sigma$- additivity}
            \sum_{n = 1}^{\infty} \mu \bigl( E_n \bigr) = \mu \bigl( E\bigr) + \mu \bigl( F \bigr)
        \end{align*}
        \item[ii)] Recall that if $ E,F \in \Mdu \Rightarrow E \setminus F \in \Mdu $ 
        \begin{align*}
            F 
            &\equalexpl{True if $ E \subset F $ }
            E \cup (F \setminus E) \Longrightarrow E \cap (F \setminus E) = \emptyset
            %\mu (F) = \mu (E) + \mu(F \setminus E)
        \end{align*}
        If we now apply (i) we get
        \begin{align*}
            \mu (F) = \mu (E) + \mu(F \setminus E) \Longrightarrow \mu (F \setminus E) 
            &\equalexpl{We can do this because $ \mu (E) < \infty$}
            \mu(F) - \mu(E)
        \end{align*}
        \item[iii)] If $ \mu (F \setminus E) < \infty $ then we can use (ii), otherwise ? (Given as exercise)
    \end{itemize}
\end{proof}

\begin{thm}[Continuity along monotone sequences]$\\$
    Let $(X,\Mdu , \mu)$ be our measure space.\\
    \begin{itemize}
        \item[i)] If $\gr{E_n}_{n \in \mathbb{N}} \subset \Mdu \ \land \ \gr{E_m} \nearrow \ \land \ E:= \lim_n E_n = \bigcup_{n=1}^{\infty} E_n $ : \\        
        \begin{equation*}
            \mu ( E) = \lim_{n} \mu (E_n)
        \end{equation*} 
        \item[ii)] if $\gr{E_n}_{n \in \mathbb{N}} \subset \Mdu \ \land \ \gr{E_m} \searrow  \ \land  \ E:= \lim_n E_n = \bigcap_{n=1}^{\infty} E_n $ : \\
        \begin{equation*}
            \mu ( E) = \lim_{n} \mu (E_n)
        \end{equation*} 
    \end{itemize}
\end{thm}
\newpage
\begin{proof}[]$\\$
    We'll use $\sigma$-additivty and the previous theorem by creating a disjoint family out of our $\gr{E_n}$ (you can check that $\sigma$-additivity is applicable yourself).\newline
    \begin{itemize}
        \item[i)] We define the following disjoint family:
        \begin{equation*}
            F_n = E_n \setminus E_{n-1} \qquad \text{where $E_0 = \emptyset$ or $F_1 = E_1$}
        \end{equation*}
        We can now start to apply some of the previously proven properties:
        \begin{align*}
            \mu (E) = \mu (\bigcup_{n=1}^{\infty} F_n)
            \equalexpl{$\sigma$-additivity}
            \sum_{n=1}^{\infty} \mu (F_n) = \sum_{n=1}^{\infty} \mu (E_n \setminus F_n)
            \equalexpl{excision}
            \sum_{n=1}^{\infty} \mu (E_n) - \mu (F_n)
            \equalexpl{telescopic series}
            \lim_{n\longrightarrow \infty} \mu(E_n) - \mu(E_0)
        \end{align*}
        \item[ii)] We define another disjoint family:
        \begin{equation*}
            G_n = E_1 \setminus E_n
        \end{equation*}
        The following properties are true:
            \begin{itemize}
                \item[0)] $G_n \in \Mdu$
                \item[1)] $\gr{G_n} \nearrow$
                \item[2)] $\bigcup_{n=1}^{\infty} G_n = E_1 \setminus \bigcap_{n=1}^{\infty} E_n = E_1 \setminus E$
            \end{itemize}
        \begin{align*}
            \equalexpl{2)}
            \mu \bigl(\bigcup_{n=1}^{\infty} G_n \bigr)
            \equalexpl{1)}
            \lim_{n\longrightarrow +\infty} \mu \bigl( G_n \bigr) = \lim_{n\longrightarrow +\infty} \bigl( \mu ( E_1) - \mu ( E_n) \bigr) = \mu \bigl( E_1 \bigr) - \lim_{n\longrightarrow +\infty} \bigl(  \mu ( E_n) \bigr)
        \end{align*}
        Therefore, since $\mu(E_1) < \infty$ we can subtract it from both sides and get:
        \begin{equation*}
            \mu(E) = \lim_{n\longrightarrow +\infty} \bigl(  \mu ( E_n) \bigr)
        \end{equation*}
    \end{itemize}
    
\end{proof}
\begin{marker}
Note that the assumption that $\mu(E_1) < \infty$ is essnetial for (ii):\newline
Let $(\mathbb{N},\Peu(\mathbb{N}),\mu_\#)$ be our measure space, take $E_n = \gr{k \in \mathbb{N} : k \geq n}$ therefore $E_n \supset E_{n+1}$\newline
Moreover $\mu_{\#} (E_n) = +\infty$ and $E = \lim_n E_n = \bigcup_{n=1}^{\infty} E_n = \emptyset$ \newline
We now observe that $\forall m \in \mathbb{N} , m \notin E_{m+1}$, in other words the limit doesn't properly converge:
\begin{equation*}
    (\mu_{\#} (E) = 0 \neq \lim_n \mu_{\#}(E_n))
\end{equation*}
\end{marker}

\begin{thm}[$\sigma$-sub additivity]$\\$
    Take $\gr{E_n}_{n \in \mathbb{N}} \subset \Mdu$ sequence of measurable sets (not necessarily disjoint), the following is true:
    \begin{equation*}
        \mu \bigl( \bigcup_{n=1}^{\infty} E_n \bigr) \leq \sum_{n=1}^{\infty} \mu \bigl( E_n \bigr)
    \end{equation*}
\end{thm}
\newpage
\begin{proof}[]$\\$
    Define the following family of sets:
    \begin{equation*}
        \begin{cases*}
            F_1 = E_1 & \\
            F_n = E_n \setminus \bigl( \bigcup_{k=1}^{n*1} E_k \bigr) & $\forall n \geq 2$
        \end{cases*}
    \end{equation*}
    These are measurable, disjoint and:
    \begin{equation*}
        \bigcup_{n=1}^{\infty} \bigl( F_n \bigr) = \bigcup_{n=1}^{\infty} \bigl( E_n \bigr)
    \end{equation*}
    Therefore we can apply the following properties:
    \begin{align*}
        \mu \bigl( \bigcup_{n=1}^{\infty} E_n \bigr) = \mu \bigl( \bigcup_{n=1}^{\infty} F_n \bigr)
        \equalexpl{$\sigma$-add}
        \sum_{n=1}^{\infty} \mu \bigl( E_n \bigr)
        \leqexpl{monotonicity $(F_n \subset E_n) \ \forall n$}
        \sum_{n=1}^{\infty} \mu \bigl( E_n \bigr)
    \end{align*}
\end{proof}
\begin{marker}[]$\\$
    If $f \colon \Mdu \leftarrow \bigl[ 0, +\infty \bigr]$ is finitely-additive and $\sigma$-subadditive then it is also $\sigma$-additive.\newline
    We'll seldom use this fact , but it is important to know that these are all linked properties. This is also how we can easily prove that something is a measure. (What's missing?)
\end{marker}
% section properties-of-measures (end)  
\section{Sets of zero measure/negligible sets} % (fold)
\label{sec:sets-of-zero-measure-negligible-sets}
Let $(X,\Mdu,\mu)$ be our measure space.
\begin{defn}[Negligible sets]$\\$
    \begin{itemize}
        \item N $in \Mdu$ has \emph{zero measure} if $\mu \bigl( N \bigr) = 0$
        \item F $\subset X$ (not necessarily in $Mdu$) is \emph{negligible} if $F \subset N$ where $\mu \bigl( N \bigr) = 0$.\newline
        Negligible sets are not generally measurable, but they have zero measure when they are (for monotonicity).
    \end{itemize}
\end{defn}

\begin{defn}[Complete measures]$\\$
    $\mu$ is \emph{complete} (or generally the measure space $(X,\Mdu,\mu)$ is complete) if every negligible set belongs to $\Mdu$
\end{defn}
A measure space $(X,\Mdu,\mu)$ needs not to be complete.\newline
Define, however the following family of sets:
\begin{equation*}
    \overline{\Mdu} = \gr{E \subset X \colon \exists F_1,F_2 \in \Mdu \text{ s.t.} \ \ F_1 \subset E \subset F_2 \ \land \ \mu \bigl( F_2 \setminus F_1 \bigr) = 0}
\end{equation*}
$\overline{\Mdu}$ is a $\sigma$-algebra.\newline
Moreover, for every set $E \in \overline{\Mdu}$ and $F_1,F_2$ as above define:
\begin{equation*}
    \overline{\mu} \bigl( E \bigr) \colon = \mu \bigl( F_1 \bigr) = \mu \bigl( F_2 \bigr)
\end{equation*}
We won't prove this fact but we'll make use of it.
\newpage
We'd like to define a measure $\lambda$ "on $\mathbb{R}$ (and subsequently $\mathbb{R}^2$,$\mathbb{R}^3$... ) such that:
\begin{itemize}
    \item $\lambda \bigl( (a,b) \bigr) = b-a$
    \item $\lambda \bigl( E + x \bigr) = \lambda \bigl( E \bigr)$
\end{itemize}
Where $a \leq b$, $-\infty \leq a < +\infty$ and $-\infty < b \leq +\infty$.\newline
In principle, we'd like to measure \emph{any} subset, that is, we want to chose $\Mdu = \Peu \bigl( \mathbb{R} \bigr)$. \\
One seemingly innocent assumption is that $\mu \bigl( \gr{a} \bigr) = 0$ where $a$ is a point, however.
\begin{thm}[Ulan]$\\$
    The only measure on $\Peu \bigl( \mathbb{R} \bigr)$ s.t. $\mu \bigl( \gr{a} \bigr) = 0$ is the trivial measure:
    \begin{equation*}
        \mu \bigl( (a,b) \bigr) = 0 
    \end{equation*}
\end{thm}
In the following, we will construct a concrete (read useful) $\sigma$-algebra which will allow the definition of the Lebesgue measure.
\begin{defn}[]$\\$
    Let $X$ be a set, an \emph{outer measure} $\mu^*$ on $X$ is a function:
    \begin{equation*}
        \mu^* \colon \Peu \bigl( x \bigr) \rightarrow [0,\infty]
    \end{equation*}
    Such that:
    \begin{enumerate}
        \item $\mu^* \bigl( \emptyset \bigr) = 0$
        \item $E \subset F \subset X \ \Longrightarrow \ \mu^* \bigl( E \bigr) \leq \mu^* \bigl( F \bigr) $ (monotonicity)
        \item $\gr{E_n}_{n} \subset \Peu(X) \ \Longrightarrow \mu^* \bigl( \bigcup_{n=1}^{\infty} E_n \bigr) \leq \sum_{n=1}^{\infty} \mu^* \bigl( E_n \bigr)$ ($\sigma$)-sub additivity
    \end{enumerate}
\end{defn}
In a way, the price to pay to define a "measure" on $\Peu(\mathbb{R})$ is to forgo $\sigma$-additivity.\\
The typical way of defining an outer measure is using a family of elementary sets: (e.g. the intervals).
\begin{prp}[]$\\$
Let $\Edu \subset \Peu(X)$ , $f \colon \Edu \rightarrow [0,+\infty]$.\\
Assume: $\emptyset,X \in \Edu \ , \ f(\emptyset) = 0$.\\
Then define, $\forall E \subset X$.
    \begin{equation*}
        \mu^* \bigl( E \bigr) = \inf \gr{\sum_{n=1}^{+\infty} f \bigl( A_n \bigr)\text{ , where} E \subset \bigcup_{n=1}^{\infty} S_n} \qquad A_n \in \Edu \text{ ($A_n$ is a covering of $E$)}
    \end{equation*}
Then $\mu^*$ is an outer measure on $X$.
\end{prp}
% section sets-of-zero-measure-negligible-sets (end)

% chapter measure_theory (end)


