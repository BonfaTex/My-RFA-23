%!TEX root = ../main.tex

\chapter{Measure Theory} % (fold)
\label{cha:measure_theory}
\thispagestyle{empty}

\section{Measure spaces} % (fold)
\label{sec:measure_spaces}

Let $X$ be a set.

\begin{defn}[$\sigma$-algebras]$\\$
A family $\Mc \subset \Pc (x)$ is called a $\sigma$-algebra if 
\begin{enumerate}
    \item[i)] $\emptyset \in \Mc$
    \item[ii)] $ E \subset \Mc \ \Longrightarrow \ E^c  = X\setminus E \in \Mc$
    \item[iii)] $\gr{E_n}_{n \in \mathbb{N}} \subset \Mc \Longrightarrow \bigcup_{n=1}^{\infty} E_n \in \Mc $ (infinite countable union)
\end{enumerate}
If (iii) is replaced by "$E_1, E_2 \in \Mc \Longrightarrow E_1 \bigcup E_2 \in \Mc$" then $\Mc$ is just an algebra (finite union).
\end{defn}

Trivial examples: $\Mc=\Pc(X)$ is the biggest $\sigma$-algebra, $\Mc=\gr{\varnothing,X}$ is the smallest $\sigma$-algebra.

We say that
\begin{itemize}
    \item $\Mc$ $\sigma$-algebra $\leadsto$ $(X,\Mc)$ is a \textbf{measurable space}
    \item $E\in\Mc$ are \textbf{measurable sets}
\end{itemize}

Basic properties of $\Mc$:
\begin{enumerate}
    \item $X=\varnothing^c\in\Mc$ (by (i)+(ii))
    \item $\Mc$ is an algebra ($\sigma$-alg. $\Longrightarrow$ alg. but not the viceversa)

    To prove this you can take a finite union (e.g. $E_1\cup E_2$) and then make infinite unions with $\varnothing$ to have an infinite union that still belongs to $\Mc$:
    \begin{equation*}
        E_1\cup E_2=\underbracket[0.5pt]{E_1\cup E_2\underbracket[0.5pt]{\cup\varnothing\cup ... \cup\varnothing\cup...}_{\in\Mc\text{ by (i)}}}_{\in\Mc\text{ by (iii)}}
    \end{equation*}

    \item $\gr{E_n}_n\subset\Mc\Longrightarrow \bigcap_{n\in\NN} E_n\in\Mc$
    \item $E,F\in\Mc\Longrightarrow E\setminus F \in \Mc$
\end{enumerate}

Now, we want to understand how to generate a $\sigma$-algebra.

\begin{thm}
    Take $\Sc\subset\Pc(X)$ any family. Then it is well defined $\sigma_0(\Sc)$, the $\sigma$-algebra generated by $\Sc$ (the smallest $\sigma$-algebra containing $\Sc$):
    \begin{itemize}
        \item[i)] $\sigma_0(\Sc)$ is a $\sigma$-algebra
        \item[ii)] $\Sc\subset\sigma_0(\Sc)$
        \item[iii)] if $\Mc$ is a $\sigma$-alg. and $\Sc\subset\Mc$ then $\sigma_0(\Sc)\subset\Mc$
    \end{itemize}
\end{thm}

\begin{proof}[Sketch]$\\$
We introduce a collection of collection of sets (we should be more strict: without knowing axiom choices we cannot properly prove this theorem, all we currently need is how to construct these $\sigma$-algebras):
\begin{equation*}
    \Vc=\left\{\Mc\subset\Pc(X)\,:\,\Mc\text{ is a }\sigma\text{-alg. and }\Sc\subset\Mc  \right\}
\end{equation*}
(notice that $\Vc$ is not empty since $\Pc(X)\in\Vc$)

Then $\sigma_0(\Sc)=\bigcap\gr{\Mc\,:\,\Mc\in\Vc}$ (to generate the smallest take the intersection of all).
\end{proof}

% section measure_spaces (end)

\section{Borel sets} % (fold)
\label{sec:borel_sets}

We now want to define between measurable and open sets, we do this by constructing the borel $\sigma$-algebra.
\begin{defn}[]$\\$
    Let $(X,d)$ be a metric space, so that open subsets of $X$ are defined (a topological space is enough).\newline
    Let $\Tc = \gr{E\subset X: E \text{is open}}$, the $\sigma$-algebra generated by $\Tc$, $\sigma_0(T)$ is the Borel $\sigma$-algebra of $X$.\newline
    We write $\Bc (x) = \sigma_0(T)$ , any $E\in \Bc (x)$ is a \emph{Borel set}
\end{defn}
All the following are Borel sets:
\begin{itemize}
    \item All open sets (or Borel-measurable sets)
    \item All closed sets (because the $\sigma$-algebra is closed under complements, and the complements of open sets are naturally closed sets)
    \item All countable intersections of open sets ($G_\delta$ sets)  
    \item All countable unions of closed sets  $F_\sigma$
\end{itemize}
We will deal with two main cases:
\begin{itemize}
    \item $X=\mathbb{R}$
    \item $X=\overline{\mathbb{R}} = [-\infty,\infty]$ 
\end{itemize}
Defining a measure in $\overline{\mathbb{R}}$ isn't trivial, we therefore define how to extend the following operations to $\overline{\mathbb{R}}$.\newline
\begin{defn}[]$\\$
    \textbf{Operations:}
    Let $a\in \mathbb{R}$.
    \begin{itemize}
        \item $a>0\ \Longrightarrow\ a\cdot \pm \infty = \pm \infty$
        \item $a<0\ \Longrightarrow\ a\cdot \mp \infty = \pm \infty$
        \item $a\pm \infty = \pm \infty$
        \item $0\cdot \pm \infty = 0$ (Note that we are not taking any limits in this definitions, we define it this way because we want the zero function to have a null integral over an unbounded interval)
        \item $+\infty -\infty$ is \textbf{not} defined
    \end{itemize}
    \textbf{Open intervals:}
    Any interval $\gr{(a,b):\ (a,b)\in \mathbb{R} \land a<b}$ remains open.\newline
    We also take the folloing intervals to be open:
    \begin{itemize}
        \item $[-\infty , b)$
        \item $(a,+\infty ]$
    \end{itemize}
\end{defn}
We will deal with $\Bc(\mathbb{R}) \text{ and }  \Bc(\overline{\mathbb{R}})$.\newline
\begin{prp}\leavevmode
    \begin{align}
        \Bc(\mathbb{R}) :&= \sigma_0 \left( \gr{\text{open sets}} \right) \nonumber \\
        &=\sigma_0 \left( \gr{\text{open intervals}} \right) \nonumber \\
        &=\sigma_0 \left( \gr{(a,+\infty)}\right) \label{borel_def_3}
    \end{align}
\end{prp}
This means we can get open sets and open intervals by taking complements,unions and intersections, starting from sets in the form of \ref{borel_def_3}.\newline
This is very useful for proving properties of open intervals and sets. We do this by showing they're true for half-lines and that they're closed for complements, unions and intersections.\newline
Generally, the smaller the set of generators, the easier properties are to prove.\newline
\newline
More generally:
\begin{prp}
    \begin{align*}
        \Bc(\mathbb{\overline{R}}) :&= \sigma_0 \left( \gr{\text{open sets}} \right) \\
        &=\sigma_0 \left( \gr{(a,+\infty)}\right)
    \end{align*}
    \begin{align*}
        \Bc(\mathbb{R}^n) :&= \sigma_0 \left( \gr{\text{open rectangles}} \right) \\
        &=\sigma_0 \left( \gr{\text{closed rectangles}} \right) \\
    \end{align*}
\end{prp}

% section borel_sets (end)

\section{Measures} % (fold)
\label{sec:measures}

Let $(X,\Mc)$ be a measurable space, how do we define a measure of this $\sigma$-algebra ?\newline
\begin{defn}[]$\\$
    A \emph{measure} on $\Mc$ is a function 
    \begin{equation*}
        \mu : \Mc \longrightarrow [0,\infty] \qquad \text{s.t.} \qquad
            \begin{aligned}
                &\text{i)}\ \mu = 0 \\
                &\text{ii)}\ \gr{E_n}_n \subset \Mc \ \text{(Countable pair-wise disjoint family)}
            \end{aligned}
    \end{equation*}
Given $i)$ and $ii)$ $\Longrightarrow \mu \bigl( \bigcup_{n\in \NN} E_n \bigr) = \sum_{n\in \NN} \mu \bigl( E_n\bigr)$
\end{defn}
\begin{defn}[]$\\$
    If we take $X,\Mc,\mu$ as above. $\Longrightarrow (X,\Mc,\mu)$ is a \emph{measure space}\newline
    In particular if 
    $\mu \bigl( X \bigr) = 1$, it is a \emph{probability space}\newline
\end{defn}
Some measure's names are:
\begin{itemize}
    \item if $\mu \bigl( X \bigr) = 1$, it is called a \emph{probabily measure}
    \item if $\mu \bigl( X \bigr) <\infty $ it is called a \emph{finite measure}
    \item if $\exists \gr{E_n}_n\ :\ \mu \bigl( E_n \bigr) < \infty \land X = \bigcup_n E_n$ then it is called a \emph{$\sigma$-finite measure}
\end{itemize}
\newpage
\begin{exa}[Examples of measures]$\\$
\begin{enumerate}
    \item[1)] For any $(X,\Mc)$\ $\longrightarrow$ \ $\mu \bigl( E \bigr) = 0 \qquad$(trivial example)
    \item[2)] For any $(X,\Mc)$\ $\longrightarrow$ \ $\mu \bigl( E \bigr) = +\infty \ \forall E \neq \emptyset \qquad$
    \item[3)] For $(X,\Pc (X) )\  \longrightarrow$ \begin{tabular}[t]{@{}l@{}}
        $\mu_{\#}  =$
        $\begin{cases}
            \# \gr{\text{elements of } E} & E\text{ is finite} \\
            +\infty & \text{otherwise}
        \end{cases}$
    \end{tabular}
    
    \item[4)] For any non-empty $X$ and $\Mc = \Pc (X)$, pick $x_0 \in X \ \longrightarrow$ \begin{tabular}[t]{@{}l@{}}
        $\delta_{E}  =$
        $\begin{cases}
            1 & x_0 \in X  \\
            0 & \text{otherwise}
        \end{cases}$
    \end{tabular}
\end{enumerate}
\end{exa}
% section measures (end)

% chapter measure_theory (end)




