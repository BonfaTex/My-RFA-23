%!TEX root = ../main.tex

\setcounter{chapter}{-1}
\chapter{Course structure}
\thispagestyle{empty}

This course is splitted in two parts:
\begin{enumerate}{
	\item Real Analysis $\leadsto$ measure and integration theory, in particular:
	\begin{itemize}{
		\item Collections and sequences of sets
		\item Measurable space, measure, outer measure
		\item Generation of an outer measure
		\item Carathéodory's condition, measure induced by an outer measure
		\item Lebesgue's measure on $\RR^n$
		\item Measurable functions
		\item The Lebesgue integral
		\item Abstract integration
		\item Monotone convergence theorem, Fatou's Lemma, Lebesgue's dominated convergence theorem
		\item Comparison between the Lebesgue and Riemann integrals
		\item Different types of convergence
		\item Derivative of a measure and the Radon-Nikodym theorem
		\item Product measures and the Fubini-Tonelli theorem
		\item Functions of bounded variation and absolutely continuous functions
	}
	\end{itemize}
	\item Functional Analysis $\leadsto$ infinte dimensional linear algebra, in particular:
	\begin{itemize}{
		\item Metric spaces, completeness, separability, compactness
		\item Normed spaces and Banach spaces
		\item Spaces of integrable functions
		\item Linear operators
		\item Uniform boundedness theorem, open mapping theorem, closed graph theorem
		\item Dual spaces and the Hahn-Banach theorem
		\item Reflexivity
		\item Weak and weak* convergences
		\item Banach-Alaoglu theorem
		\item Compact operators
		\item Hilbert spaces
		\item Projection theorem, Riesz representation theorem
		\item Orthonormal basis, abstract Fourier series
		\item Spectral theorem for compact symmetric operators
		\item Fredholm alternativ
	}
	\end{itemize}
}
\end{enumerate}

The foundation of this theory is the \emph{Set Theory}, that is going to be explained in the next chapter. Enjoy!

\bigskip
\bigskip

\textbf{NB:} this page will be updated with more details and maybe the list of proofs.