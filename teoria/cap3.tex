%!TEX root = ../main.tex

\chapter{Lebesgue Measure} % (fold)
\label{cha:lebesgue_measure}
\thispagestyle{empty}

\section{Towards the Lebesgue measure} % (fold)
\label{sec:towards_the_lebesgue_measure}

We'd like to define a measure $\lambda$ on $\mathbb{R}$ (and subsequently $\mathbb{R}^2$, $\mathbb{R}^3$, ...) such that:
\begin{itemize}
    \item $\lambda \big( (a,b) \big) = b-a$ $\ \forall a,b\in\overline{\RR}$, $a\leq b$ $\leadsto$ the measure of an interval is simply its lenght

    \item $\lambda \left( x+E \right) = \lambda \left( E \right)$ $\ \forall x\in \RR,\ \forall E\subset \RR$ $\leadsto $ "invariance under translations"
\end{itemize}

What about  the $\sigma$-algebra? In principle, we'd like to measure any subset, so $\Mdu = \Peu \left( \mathbb{R} \right)$. In this way we measure points of $\RR$: $\lambda \left( \gr{a} \right) = 0$. But it turns out that:

\begin{thm}[Ulam]$\\$
    The only measure on $\Peu \left( \mathbb{R} \right)$ s.t. $\lambda \left( \gr{a} \right) = 0$ is the trivial measure.
\end{thm}

Because of this, if we take $\Mdu=\Peu(\RR)$ then $\lambda\big( (a,b) \big) = 0 $ for any $(a,b)$ (because $\lambda$ is the trivial measure). Hence, we need a smaller $\sigma$-algebra.

% section towards_the_lebesgue_measure (end)

\section{Outer measures} % (fold)
\label{sec:outer_measures}

In the following, we build a concrete $\sigma$-algebra which allows the definition of the Lebesgue measure.

\begin{defn}[Outer measure]$\\$
Let $X$ be a set. An outer measure $\mu^*$ on $X$ is a function $\mu^* : \Peu (X) \to [0,+\infty]$ such that:
\begin{enumerate}[(1)]
\item $\mu^* \left( \emptyset \right) = 0$
\item Monotonicity: $E \subset F \subset X \ \Longrightarrow \ \mu^* \left( E \right) \leq \mu^* \left( F \right) $
\item $\sigma$-sub additivity: $\gr{E_n}_{n} \subset \Peu(X) \ \Longrightarrow \mu^* \left(\displaystyle \bigcup_{n=1}^{\infty} E_n \right) \leq \displaystyle\sum_{n=1}^{\infty} \mu^* \left( E_n \right)$
\end{enumerate}
\end{defn}

With the outer measure we are not fixing the $\sigma$-algebra, but the price to pay is to forgo $\sigma$-additivity.

\textit{Examples.} $\mu^*:\Peu(\RR)\to[0,\infty]$ s.t.
\begin{equation*}
\mu^*(A)=\begin{cases}
    0, &\text{ if }A=\varnothing\\
    1, &\text{otherwise}
\end{cases}
\end{equation*}

is an outer measure, but \underline{not} a measure. Instead, $\mu^*:\Peu(\NN)\to[0,\infty]$ s.t. 
\begin{equation*}
\mu^*(A)=\begin{cases}
    |A|, &\text{ if }A\text{ finte}\\
    \infty, &\text{otherwise}
\end{cases}
\end{equation*}
is an outer measure and a measure too (indeed, it's the counting measure).

\newpage

Those were trivial examples. What we really are interested in is:

$\boxed{\text{Q}_1}$ : how to build an outer measure?

$\boxed{\text{Q}_2}$ : how to go from an outer measure to a measure, i.e., how to gain $\sigma$-additivity?

\begin{rem}
$\mu$ measure implies $\mu$ outer measure (because $\sigma$-add. implies $\sigma$-subadd.), but the counter isn't true: remember Ulam, we have to work on a smaller collection of subsets than $\Peu(X)$.    
\end{rem}

\bigskip

$\boxed{\text{A}_1}$

The most common way to obtain outer measures is to start with a family $\bondox{E}$ of \emph{elementary sets} on which a notion of measure is defined (e.g., intervals and their lengths, rectangles and their areas, etc.) and then to approximate arbitrary sets "from the outside" by countable unions of members of $\bondox{E}$. The precise construction is as follows.

\begin{proposition}[Generation of an outer measure from elementary sets]$\\$
Let $\bondox{E}\subset\Peu(X)$ and $f:\bondox{E}\to[0,\infty]$ be such that $\varnothing,X\in\bondox{E}$ and $f(\varnothing)=0$. For any $E\subset X$, define
\begin{equation}
\label{outer_inf}
\begin{aligned}
\mu^*(E)&=\inf\left\{\sum^{\infty}_{n=1}f\left( C_n \right)\,:\,C_n\in\bondox{E}\text{ and }E\subset\bigcup_{n=1}^{\infty}C_n \right\} \\
&=\inf\left\{\sum^{\infty}_{n=1}f\left( C_n \right)\,:\,\gr{C_n}\subset\bondox{E}\text{ is a covering of }E \right\}
\end{aligned}
\end{equation}

Then $\mu^*$ is an outer measure on $X$.
\end{proposition}

Few initial comments:
\begin{itemize}
    \item As we said before, you can think of $\bondox{E}$ as \emph{intervals} and $f$ as \emph{length}.

    \item Speaking of the inf:
    \begin{marker}
    Here, we want to be clear about the definition of infimum because it will be useful for both this and the next demonstrations.

    \bigskip

    Let $T\subset\RR$, $T\neq\varnothing$, $\inf T=\overline{t}\in\RR$. The \emph{definition of infimum} means that:
    \begin{itemize}[$\triangleright$]
        \item $\overline{t}\leq t\quad\forall t\in T$ (it's a lower bound)
        \item $\forall\delta>0\quad \exists t_\delta\in T\quad$s.t.$\quad t_\delta\leq \overline{t}+\delta$ (it cannot be better)
    \end{itemize}
    So the inf is the best/greatest lower bound.

    \bigskip

    \textbf{NB:} l'inf non deve per forza appartenere a $T$, e.g. $\inf\left\{ (0,1) \right\}=0$. Infatti la prima proprietà dice che deve essere più piccolo di ogni elemento di $T$, invece la seconda dice che è il "migliore" nel senso che ogni volta che fissi un $\delta$ allora esisterà sempre un'elemento nell'insieme compreso tra l'inf e l'inf+$\delta$; se così non fosse, allora ogni elemento di $T$ sarebbe $\geq$ dell'inf+$\delta$, e di conseguenza l'inf sarebbe in realtà inf+$\delta$.

    \bigskip

    The most practical way to use this is to construct a "Minimizing sequence" by applying repetitively the second propertry for different values of $\delta$ (tipically $\delta=\delta_n\xrightarrow{n\to\infty}0$). For example:
    \begin{equation*}
    \delta=\frac{1}{n}\ (n\geq 1)\quad\leadsto\quad t_n\in T\quad\text{s.t.}\quad \overline{t}\leq t_n\leq \overline{t}+\frac{1}{n} 
    \end{equation*}
    \end{marker}

    In our case the set
    \begin{equation*}
    \left\{\sum^{\infty}_{n=1}f\left( C_n \right)\,:\,C_n\in\bondox{E}\text{ and }E\subset\bigcup_{n=1}^{\infty}C_n \right\}
    \end{equation*}
    cannot be empty $\big($cause $\{X,\varnothing,\varnothing,...\}$ is a covering of $E$$\big)$ thus the inf is well defined.
\end{itemize}

\newpage

\begin{proof}$\\$
We want to show that $\mu^*$ defined by (\ref{outer_inf}) is an outer measure, hence we have to prove the points in its definition:
\begin{itemize}
    \item[(0)] $\mu^*:\Peu(X)\to[0,\infty]$ ?

    Take a generic $A\subset\Peu(X)$. Then $\{X,\varnothing,\varnothing,...\}$ is a cover of $A$ and so the inf is well defined, i.e. $\geq 0$.

    \item[(1)] $\mu^*(\varnothing)=0$ ?

    Since $\{\varnothing,\varnothing,...\}$ is a covering of $\varnothing$ we have
    \begin{equation*}
    0 \overset{(0)}{\leq}\mu^*(\varnothing)\leq\sum_{n=1}^\infty f(\varnothing)=0\qquad\Longrightarrow\qquad \mu*(\varnothing)=0
    \end{equation*}

    \item[(2)] monotonicity?

    Let's take $A\subset B\subset X$ and write down the definition of $\mu^*$ for the bigger set:
    \begin{equation*}
    \begin{aligned}
    \mu^*(B)&=\inf\left\{\sum^{\infty}_{n=1}f\left( C_n \right)\,:\,C_n\in\bondox{E}\text{ and }B\subset\bigcup_{n=1}^{\infty}C_n \right\} \\
    &=\left|
    \begin{aligned}
    A\subset B\ &\Rightarrow\ \bigcup_{n=1}^{\infty}C_n\text{ is also a covering for }A \\
    &\Rightarrow\text{inf of all coverings can't be bigger}
    \end{aligned} 
    \right. \\
    &\geq \inf\left\{\sum^{\infty}_{n=1}f\left( C_n \right)\,:\,C_n\in\bondox{E}\text{ and }A\subset\bigcup_{n=1}^{\infty}C_n \right\} = \mu*(A) \qquad \Longrightarrow\qquad \mu^*(A)\leq\mu^*(B)
    \end{aligned}
    \end{equation*}

    \item[(3)] $\sigma$-subadditivity?

    Take $\left\{ A_n \right\}_{n\in\NN}\subset\Peu(X)$. We have to show that
    \begin{equation*}
    \mu^*\left( \bigcup_{n\in\NN}A_n \right)\leq \sum_{n=1}^\infty \mu^*\left( A_n \right)
    \end{equation*}

    Suppose $\mu^*\left( A_n \right)<\infty\ \forall n$ (because if $\,\exists \overline{n}$ s.t. $\mu^*\left( A_{\overline{n}} \right)=\infty$ then $\mu^*\left( \bigcup_{n}A_n \right)\leq\infty$ trivial).

    Fix $\Ec>0$ and choose $\Ec_n$ s.t. $\displaystyle\sum_{n=1}^\infty\Ec_n=\Ec$ - for istance, $\Ec_n=\dfrac{\Ec}{2^n}$.    

    \textbf{For every} $\mu^*\left( A_n \right)$ we can use the definition of infimum:
    \begin{equation*}
    \exists\left\{ C_{n,k} \right\}_{k\in\NN}\subset\bondox{E}\quad\text{s.t.}\quad \underbracket[0.5pt]{A_n\subset\bigcup_{k=1}^\infty C_{n,k}}_{\circleddash}\quad\text{and}\quad \underbracket[0.5pt]{\mu^*\left( A_n \right)\geq \sum_{k=1}^\infty f\left( C_{n,k} \right)-\Ec_n}_{\text{inf=largest lower bound }\circledast}
    \end{equation*}

    Now, thanks to $\circleddash$
    \begin{equation*}
    \bigcup_n A_n \subset \bigcup_n \Bigg( \bigcup_k C_{n,k} \Bigg)=\bigcup_{n,k} C_{n,k}\qquad\text{(countable union)}
    \end{equation*}

    thus
    \begin{equation*}
    \mu^*\left( \bigcup_{n\in\NN}A_n \right)\overset{\text{mon.}}{\leq} \mu^*\left( \bigcup_{n,k}C_{n,k} \right) \overset{\text{def. }\mu^*}{\leq} \sum_n\Bigg(\sum_k f\left( C_{n,k} \right) \Bigg) \overset{\circledast}{\leq} \sum_{n} \big( \mu^*\left( A_n \right)+\Ec_n \big) =  \sum_{n} \mu^*\left( A_n \right) + \Ec 
    \end{equation*}

    Since $\Ec$ is arbitrary, we're done.
\end{itemize}
\end{proof}

% ... incompleto

% chapter lebesgue_measure (end)